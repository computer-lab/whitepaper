\documentclass[12pt]{article}
\usepackage[margin=1in]{geometry}
\setlength{\parindent}{1cm}
\linespread{1.2}
\usepackage{indentfirst}
\title{Computer Lab: A Lightweight Software Consultancy}
\author{Patrick Steadman\thanks{Thanks to Alice Aliceson, Bob Bobson...}}
\date{June 22nd, 2017}
\begin{document}
\maketitle

\begin{abstract}
We think that software consultancies can operate excellently in a very
lightweight fashion, while still delivering deep domain expertise that rivals
the value provided by prestigious management consultancies like McKinsey or
Bain. We believe that developments in communications, organizational thought,
and payment systems will allow businesses to exist without some of the
political, bureaucratic and financial overhead that is often consider an
essential or even desireable aspect of 'startup culture' during the past few
decades. We believe that Computer Lab's view of the role of the programmer will
result in both more fulfilled programmers and better software, and a more
healthy relationship between busineses and software.
\end{abstract}

\section{Introduction}

Why does a consultancy need a whitepaper?

Wikipedia defines a whitepaper as a 'an authoritative report or guide that
informs readers concisely about a complex issue and presents the issuing body's
philosophy on the matter'.

Recently, in the crypto space, whitepapers are often used to introduce a novel
technical approach or project.

This document does two things: it introduces some of the philosophical ideas
behind Computer Lab, most of which are already best practices, and then details
two areas where we believe Computer Lab has a novel approach.

We think that software consultancies can operate excellently in a very
lightweight fashion, while still delivering deep domain expertise that rivals
the value provided by prestigious management consultancies like McKinsey or
Bain. We believe that developments in communications, organizational thought,
and payment systems will allow businesses to exist without some of the
political, bureaucratic and financial overhead that is often consider an
essential or even desireable aspect of 'startup culture' during the past few
decades. We believe that Computer Lab's view of the role of the programmer will
result in both more fulfilled programmers and better software, and a more
healthy relationship between busineses and software.

While we hope for this document to be somewhat 'authorative', Computer Lab is
designed around the idea that most private businesses are lifestyle businesses:
the happiness of the owners and stakeholders is more important than any business
strategy or metric.  Often, the concept of 'strategy' is abused for politcal
reasons, or to manipulate workers.  This document is less of a plan, and more of
an expression of our aspirations and best intentions.  Without a clear statement
of goals, a consultancy is at risk of being yanked around by clients or peer
pressure.

\section{Fundamentals}
\subsection{Context}

Computer Lab, as a company, arose somewhat spontaneously out of a group DM on
Twitter and then a Slack.

\subsection{Goals}

- domain-driven


\subsection{Organization}

There are three primary roles in the Computer Lab organization: \textbf{Partner},
\textbf{Member}, and \textbf{Guest}.

A \textbf{Partner} owns a part of the Cloister Products LLC, the Limited Liability
Company that backs Computer Lab.  A Partner can make deals on behalf of the
company, and recruit Lab members to work on these projects.  Computer Lab's
profits are periodically distributed to partners based on an "Eat What You Kill"
formula that includes the business generated and managed by each partner, each
partner's personal billable work, and other contributions as agreed on by the
partners.  The partners are expected to create a satisfactory arrangement among
themselves.

A \textbf{Member} has full access to the Computer Lab Slack, GitHub, and other
platforms where projects are discussed and executed.  They can choose to work on
projects managed by the Partners.  If they do so, a transparent "cut" of the
value of their work is retained by the company.  Currently the benchmark for
this cut is 25\%, assuming that the Member is an independent contractor.  The
network and tools provided by Computer Lab should help the Member make a living
as a software development contractor/consultant.

How is one onboarded as memeber...I guess adding someone to Bonsai Talent
Network.

A \textbf{Guest} is someone that the company is working with, either as a client,
contractor, or employee, who is given some access to Computer Lab systems in
order to facilitate a project or goal.  An example of this is someone who is
added to the Slack as a single-channel guest for the duration of a project.

Partners, Members or Guests may all receive salary and benefits from
the LLC as necessary or desired.

\subsection{Key Processes}
Every business has a set of key processes that seek to maintain a set of
invariants (or good states at certain times). For example, one key invariant of
a business is that all its employees are paid they salaries that they are owed
on paydays. Many complex systems arise to ensure this simple invariant.
If these invariants are not maintained, higher-level strategy and quality work
will be undermined. Just as with code, having strong invariants in your business
makes it easier to reason about.

The invariants we don't try to maintain are important too. Not everything needs
to be codified into a process.

Here are a list of key proccess essential to Computer Lab. Some of our
approaches to these processes are more novel, and some are quite standard.
However, whether our approach is novel or an industry best practice, all of
these processes are critical.

\subsubsection{Accounting and Contracting}
bonsai
xero

\subsubsection{Communications}
slack gmail office events

\subsubsection{CRM and Deal Tracking}

\section{Domains}
\subsection{Arts}
\subsection{Life Sciences}
\subsection{Rapid MVPs}
\section{Current Status}
\section{Roadmap}
\section{Conclusion}


\end{document}
